\documentclass[11pt,letterpaper]{article}
\usepackage[utf8]{inputenc}
\usepackage[left=1in,right=1in,top=1in,bottom=1in]{geometry}
\usepackage{amsfonts,amsmath}
\usepackage{graphicx,float}
% -----------------------------------
\usepackage{hyperref}
\hypersetup{%
  colorlinks=true,
  linkcolor=blue,
  citecolor=blue,
  urlcolor=blue,
  linkbordercolor={0 0 1}
}
% -----------------------------------
\usepackage[style=authoryear-icomp,backend=biber]{biblatex}
\addbibresource{citation.bib}
% -----------------------------------
\usepackage{fancyhdr}
\newcommand\course{MATH-UA.0325\\Analysis}
\newcommand\hwnumber{2}                  % <-- homework number
\newcommand\NetIDa{Ryan Sh\`iji\'e D\`u} 
\newcommand\NetIDb{September 19th, 2024}
\pagestyle{fancyplain}
\headheight 35pt
\lhead{\NetIDa\\\NetIDb}
\chead{\textbf{\Large Worksheet \hwnumber}}
\rhead{\course}
\lfoot{}
\cfoot{}
\rfoot{\small\thepage}
\headsep 1.5em
% -----------------------------------
\usepackage{titlesec}
\renewcommand\thesubsection{(\arabic{section}.\alph{subsection})}
\titleformat{\subsection}[runin]
        {\normalfont\bfseries}
        {\thesubsection}% the label and number
        {0.5em}% space between label/number and subsection title
        {}% formatting commands applied just to subsection title
        []% punctuation or other commands following subsection title
% -----------------------------------
\setlength{\parindent}{0.0in}
\setlength{\parskip}{0.1in}
% -----------------------------------
\newcommand{\de}{\mathrm{d}}
\newcommand{\DD}{\mathrm{D}}
\newcommand{\pe}{\partial}
\newcommand{\mcal}{\mathcal}
%\newcommand{\pdx}{\left|\frac{\partial}{\partial_x}\right|}

\newcommand{\dsp}{\displaystyle}

\newcommand{\norm}[1]{\left\Vert #1 \right\Vert}
%\newcommand{\mean}[1]{\left\langle #1 \right\rangle}
\newcommand{\mean}[1]{\overline{#1}}
\newcommand{\inner}[2]{\left\langle #1,#2\right\rangle}

\newcommand{\ve}[1]{\boldsymbol{#1}}

\newcommand{\thus}{\Rightarrow \quad }
\newcommand{\fff}{\iff\quad}
\newcommand{\qdt}[1]{\quad \mbox{#1} \quad}

\renewcommand{\Re}{\mathrm{Re}}
\renewcommand{\Im}{\mathrm{Im}}
\newcommand{\E}{\mathbb{E}}
\newcommand{\lap} {\nabla^2}
\renewcommand{\div}{\nabla\cdot}

\newcommand{\csch}{\text{csch}}
\newcommand{\sech}{\text{sech}}


\newcommand{\hot}{\text{h.o.t.}}

\newcommand{\ssp}{\left.\qquad\right.}

\newcommand{\var}{\text{var}}
\newcommand{\cov}{\text{cov}}

%%%%%%%%%%%%%%%%%%%%%%%%%%%%%%%%%%%%%%%%%%%%%%%%%%
\makeatletter
\newcommand*{\mint}[1]{%
  % #1: overlay symbol
  \mint@l{#1}{}%
}
\newcommand*{\mint@l}[2]{%
  % #1: overlay symbol
  % #2: limits
  \@ifnextchar\limits{%
    \mint@l{#1}%
  }{%
    \@ifnextchar\nolimits{%
      \mint@l{#1}%
    }{%
      \@ifnextchar\displaylimits{%
        \mint@l{#1}%
      }{%
        \mint@s{#2}{#1}%
      }%
    }%
  }%
}
\newcommand*{\mint@s}[2]{%
  % #1: limits
  % #2: overlay symbol
  \@ifnextchar_{%
    \mint@sub{#1}{#2}%
  }{%
    \@ifnextchar^{%
      \mint@sup{#1}{#2}%
    }{%
      \mint@{#1}{#2}{}{}%
    }%
  }%
}
\def\mint@sub#1#2_#3{%
  \@ifnextchar^{%
    \mint@sub@sup{#1}{#2}{#3}%
  }{%
    \mint@{#1}{#2}{#3}{}%
  }%
}
\def\mint@sup#1#2^#3{%
  \@ifnextchar_{%
    \mint@sup@sub{#1}{#2}{#3}%
  }{%
    \mint@{#1}{#2}{}{#3}%
  }%
}
\def\mint@sub@sup#1#2#3^#4{%
  \mint@{#1}{#2}{#3}{#4}%
}
\def\mint@sup@sub#1#2#3_#4{%
  \mint@{#1}{#2}{#4}{#3}%
}
\newcommand*{\mint@}[4]{%
  % #1: \limits, \nolimits, \displaylimits
  % #2: overlay symbol: -, =, ...
  % #3: subscript
  % #4: superscript
  \mathop{}%
  \mkern-\thinmuskip
  \mathchoice{%
    \mint@@{#1}{#2}{#3}{#4}%
        \displaystyle\textstyle\scriptstyle
  }{%
    \mint@@{#1}{#2}{#3}{#4}%
        \textstyle\scriptstyle\scriptstyle
  }{%
    \mint@@{#1}{#2}{#3}{#4}%
        \scriptstyle\scriptscriptstyle\scriptscriptstyle
  }{%
    \mint@@{#1}{#2}{#3}{#4}%
        \scriptscriptstyle\scriptscriptstyle\scriptscriptstyle
  }%
  \mkern-\thinmuskip
  \int#1%
  \ifx\\#3\\\else_{#3}\fi
  \ifx\\#4\\\else^{#4}\fi  
}
\newcommand*{\mint@@}[7]{%
  % #1: limits
  % #2: overlay symbol
  % #3: subscript
  % #4: superscript
  % #5: math style
  % #6: math style for overlay symbol
  % #7: math style for subscript/superscript
  \begingroup
    \sbox0{$#5\int\m@th$}%
    \sbox2{$#5\int_{}\m@th$}%
    \dimen2=\wd0 %
    % => \dimen2 = width of \int
    \let\mint@limits=#1\relax
    \ifx\mint@limits\relax
      \sbox4{$#5\int_{\kern1sp}^{\kern1sp}\m@th$}%
      \ifdim\wd4>\wd2 %
        \let\mint@limits=\nolimits
      \else
        \let\mint@limits=\limits
      \fi
    \fi
    \ifx\mint@limits\displaylimits
      \ifx#5\displaystyle
        \let\mint@limits=\limits
      \fi
    \fi
    \ifx\mint@limits\limits
      \sbox0{$#7#3\m@th$}%
      \sbox2{$#7#4\m@th$}%
      \ifdim\wd0>\dimen2 %
        \dimen2=\wd0 %
      \fi
      \ifdim\wd2>\dimen2 %
        \dimen2=\wd2 %
      \fi
    \fi
    \rlap{%
      $#5%
        \vcenter{%
          \hbox to\dimen2{%
            \hss
            $#6{#2}\m@th$%
            \hss
          }%
        }%
      $%
    }%
  \endgroup
}

\begin{document}

% \section{Defining the rationals}
% [5.4.2 of \cite{Hajlasz_18}] Take the set $\mcal{R} = \{(a,b), a,b\in\mathbb{Z}, b\neq 0\}$. For $(a,b)$ and $(c,d)\in\mcal{R}$, we write $(a,b)\sim(c,d)$ if $ad=bc$. Check if this is an equivalence relation.

% \section{}
% Let $(F,+,\cdot,<)$ be an ordered field, show

% \subsection{}
% Take $a\in F$ such that $a>0$, then 
% \begin{align}
%     a^{-1}>0.
% \end{align}

% \subsection{}
% Take $a,b\in F$ such that $0<a<b$, then 
% \begin{align}
%     0<b^{-1}<a^{-1}.
% \end{align}

% \subsection{}
% Take $a,b,c\in F$, show that 
% \begin{align}
%     2ab \leq a^2+b^2
% \end{align}
% and
% \begin{align}
%     ab+bc+ac \leq a^2+b^2+c^2.
% \end{align} 

\section{}
Let $S$ be a non-empty bounded subset of $\mathbb{R}$.
\subsection{} Prove that $\inf S\leq \sup S$.
\subsection{} What can you say about $S$ if $\inf S= \sup S$.

\section{}
Let $A$ be a non-empty subset of $\mathbb{R}$ which is bounded below and let
\begin{align}
    -A = \{-a:a\in A\}.
\end{align}
Prove that $\inf A = -\sup(-A)$.

\section{}
Let $S$ and $T$ be two non-empty bounded subsets of $\mathbb{R}$ and let

\subsection{} [1.1.4 of \cite{Lebl_23}] Prove that if $S \subset T$, then $\inf T \leq \inf S \leq \sup S \leq \sup T$.

\subsection{} Prove that $\sup(S\cup T) = \max\{\sup(S),\sup(T)\}$. 

% \section{}
% [1.2.9 of \cite{Lebl_23}] Let $A$ and $B$ be two non-empty bounded subsets of $\mathbb{R}$ and let
% \begin{align}
%     S = \{a+b:a\in A \text{ and } b\in B\}.
% \end{align}
% \subsection{} Prove that $\sup S = \sup A+\sup B$.
% \subsection{} Prove that $\inf S = \inf A+\inf B$.

\section{}
Let $F\subset \mathbb{R}$ be a closed set bounded above. Prove that $\sup F\in F$.

\section{}
\subsection{} Show that if a sequence $\{a_n\}_{n\in\mathbb{N}}$ of real numbers converges to $a$, then the sequence $\{|a_n|\}_{n\in\mathbb{N}}$ converges to $|a|$. 
\subsection{} Show (via an example) that the converse is not true.

\section{}
Let's work with the vector space $\mathbb{R}^n$ and $A$ a non-empty subset of $\mathbb{R}^n$. Prove that $A$ is open if and only if it can be written as the union of a family of open balls of the form
\begin{align}
    B_r(x) = \{y\in \mathbb{R}^n: \norm{x-y}_2< r\}. 
\end{align}


\vfill
\printbibliography


\end{document}