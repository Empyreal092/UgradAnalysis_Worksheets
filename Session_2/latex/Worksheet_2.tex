\documentclass[11pt,letterpaper]{article}
\usepackage[utf8]{inputenc}
\usepackage[left=1in,right=1in,top=1in,bottom=1in]{geometry}
\usepackage{amsfonts,amsmath}
\usepackage{graphicx,float}
% -----------------------------------
\usepackage{hyperref}
\hypersetup{%
  colorlinks=true,
  linkcolor=blue,
  citecolor=blue,
  urlcolor=blue,
  linkbordercolor={0 0 1}
}
% -----------------------------------
\usepackage[style=authoryear-icomp,backend=biber]{biblatex}
\addbibresource{citation.bib}
% -----------------------------------
\usepackage{fancyhdr}
\newcommand\course{MATH-UA.0325\\Analysis}
\newcommand\hwnumber{2}                  % <-- homework number
\newcommand\NetIDa{Ryan Sh\`iji\'e D\`u} 
\newcommand\NetIDb{September 19th, 2024}
\pagestyle{fancyplain}
\headheight 35pt
\lhead{\NetIDa\\\NetIDb}
\chead{\textbf{\Large Worksheet \hwnumber}}
\rhead{\course}
\lfoot{}
\cfoot{}
\rfoot{\small\thepage}
\headsep 1.5em
% -----------------------------------
\usepackage{titlesec}
\renewcommand\thesubsection{(\arabic{section}.\alph{subsection})}
\titleformat{\subsection}[runin]
        {\normalfont\bfseries}
        {\thesubsection}% the label and number
        {0.5em}% space between label/number and subsection title
        {}% formatting commands applied just to subsection title
        []% punctuation or other commands following subsection title
% -----------------------------------
\setlength{\parindent}{0.0in}
\setlength{\parskip}{0.1in}
% -----------------------------------
\input{../../command.tex}
\begin{document}

% \section{Defining the rationals}
% [5.4.2 of \cite{Hajlasz_18}] Take the set $\mcal{R} = \{(a,b), a,b\in\mathbb{Z}, b\neq 0\}$. For $(a,b)$ and $(c,d)\in\mcal{R}$, we write $(a,b)\sim(c,d)$ if $ad=bc$. Check if this is an equivalence relation.

% \section{}
% Let $(F,+,\cdot,<)$ be an ordered field, show

% \subsection{}
% Take $a\in F$ such that $a>0$, then 
% \begin{align}
%     a^{-1}>0.
% \end{align}

% \subsection{}
% Take $a,b\in F$ such that $0<a<b$, then 
% \begin{align}
%     0<b^{-1}<a^{-1}.
% \end{align}

% \subsection{}
% Take $a,b,c\in F$, show that 
% \begin{align}
%     2ab \leq a^2+b^2
% \end{align}
% and
% \begin{align}
%     ab+bc+ac \leq a^2+b^2+c^2.
% \end{align} 

\section{}
Let $S$ be a non-empty bounded subset of $\mathbb{R}$.
\subsection{} Prove that $\inf S\leq \sup S$.
\subsection{} What can you say about $S$ if $\inf S= \sup S$.

\section{}
Let $A$ be a non-empty subset of $\mathbb{R}$ which is bounded below and let
\begin{align}
    -A = \{-a:a\in A\}.
\end{align}
Prove that $\inf A = -\sup(-A)$.

\section{}
Let $S$ and $T$ be two non-empty bounded subsets of $\mathbb{R}$ and let

\subsection{} [1.1.4 of \cite{Lebl_23}] Prove that if $S \subset T$, then $\inf T \leq \inf S \leq \sup S \leq \sup T$.

\subsection{} Prove that $\sup(S\cup T) = \max\{\sup(S),\sup(T)\}$. 

% \section{}
% [1.2.9 of \cite{Lebl_23}] Let $A$ and $B$ be two non-empty bounded subsets of $\mathbb{R}$ and let
% \begin{align}
%     S = \{a+b:a\in A \text{ and } b\in B\}.
% \end{align}
% \subsection{} Prove that $\sup S = \sup A+\sup B$.
% \subsection{} Prove that $\inf S = \inf A+\inf B$.

\section{}
Let $F\subset \mathbb{R}$ be a closed set bounded above. Prove that $\sup F\in F$.

\section{}
Let's work with the vector space $\mathbb{R}^n$ and $A$ a non-empty subset of $\mathbb{R}^n$. Prove that $A$ is open if and only if it can be written as the union of a family of open balls of the form
\begin{align}
    B_r(x) = \{y\in \mathbb{R}^n: \norm{x-y}_2< r\}. 
\end{align}

\newpage
\section{}
\subsection{} Show that if a sequence $\{a_n\}_{n\in\mathbb{N}}$ of real numbers converges to $a$, then the sequence $\{|a_n|\}_{n\in\mathbb{N}}$ converges to $|a|$. 
\subsection{} Show (via an example) that the converse is not true.


\vfill
\printbibliography


\end{document}