\documentclass[11pt,letterpaper]{article}
\usepackage[utf8]{inputenc}
\usepackage[left=1in,right=1in,top=1in,bottom=1in]{geometry}
\usepackage{amsfonts,amsmath,amsthm}
\usepackage{graphicx,float}
% -----------------------------------
\usepackage{hyperref}
\hypersetup{%
  colorlinks=true,
  linkcolor=blue,
  citecolor=blue,
  urlcolor=blue,
  linkbordercolor={0 0 1}
}
% -----------------------------------
\usepackage[style=authoryear-icomp,backend=biber]{biblatex}
\addbibresource{citation.bib}
% -----------------------------------
\usepackage{fancyhdr}
\newcommand\course{MATH-UA.0325\\Analysis}
\newcommand\hwnumber{5 Partial Solutions}                  % <-- homework number
\newcommand\NetIDa{Ryan Sh\`iji\'e D\`u} 
\newcommand\NetIDb{March 1st, 2024}
\pagestyle{fancyplain}
\headheight 35pt
\lhead{\NetIDa\\\NetIDb}
\chead{\textbf{\Large Worksheet \hwnumber}}
\rhead{\course}
\lfoot{}
\cfoot{}
\rfoot{\small\thepage}
\headsep 1.5em
% -----------------------------------
\usepackage{titlesec}
\renewcommand\thesubsection{(\arabic{section}.\alph{subsection})}
\titleformat{\subsection}[runin]
        {\normalfont\bfseries}
        {\thesubsection}% the label and number
        {0.5em}% space between label/number and subsection title
        {}% formatting commands applied just to subsection title
        []% punctuation or other commands following subsection title
% -----------------------------------
\setlength{\parindent}{0.0in}
\setlength{\parskip}{0.1in}
% -----------------------------------
\input{../../command.tex}
\begin{document}

\setcounter{section}{3}
\section{}
Let $\{a_n\}_{n\geq 1}$ be a decreasing sequence of non-negative numbers such
that $\sum_{n\geq 1} a_n<\infty$. Show that
\begin{align}
    \lim_{n\to\infty} na_n = 0.
\end{align}

\begin{proof}
Pick any $\epsilon>0$. Because $\sum_{n\geq 1}\{a_n\}$ converges, by the Cauchy criterion, there exists $N_1\in\mathbb{N}$ such that for all $p>0$,
\begin{align}
    \sum_{N_1+1}^{N_1+p}a_n<\frac{\epsilon}{2}.
\end{align}
Because $\{a_n\}_{n\geq 1}$ is decreasing, 
\begin{align}
    pa_{N_1+p}\leq\sum_{N_1+1}^{N_1+p}a_n<\frac{\epsilon}{2}.
\end{align}

Moreover, the sequence $\{a_n\}_{n\geq 1}$ converges to 0. There exists $N_2\in\mathbb{N}$ s.t. for all $p\geq N_2$, 
\begin{align}
    &a_{N_1+p}<\frac{\epsilon}{2N_1}\\
    \thus &N_1 a_{N_1+p}<\frac{\epsilon}{2}.
\end{align}

Adding two together we have, for all $p\geq N_2$, 
\begin{align}
    (N_1+p)a_{N_1+p}<\epsilon
\end{align}
Change of variable, we have for all $m\geq N_1+N_2$, $ma_m<\epsilon$. This proves $\lim_{n\rightarrow\infty}na_n = 0$. 

\end{proof}


% \vfill
% \printbibliography


\end{document}