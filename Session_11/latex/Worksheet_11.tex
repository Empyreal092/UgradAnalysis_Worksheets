\documentclass[11pt,letterpaper]{article}
\usepackage[utf8]{inputenc}
\usepackage[left=1in,right=1in,top=1in,bottom=1in]{geometry}
\usepackage{amsfonts,amsmath,amsthm}
\usepackage{graphicx,float}
% -----------------------------------
\usepackage{hyperref}
\hypersetup{%
  colorlinks=true,
  linkcolor=blue,
  citecolor=blue,
  urlcolor=blue,
  linkbordercolor={0 0 1}
}
% -----------------------------------
\usepackage[style=authoryear,backend=biber]{biblatex}
\addbibresource{citation.bib}
% -----------------------------------
\usepackage{fancyhdr}
\newcommand\course{MATH-UA.0325\\Analysis}
\newcommand\hwnumber{11}                  % <-- homework number
\newcommand\NetIDa{Ryan Sh\`iji\'e D\`u} 
\newcommand\NetIDb{November 21st, 2024}
\pagestyle{fancyplain}
\headheight 35pt
\lhead{\NetIDa\\\NetIDb}
\chead{\textbf{\Large Worksheet \hwnumber}}
\rhead{\course}
\lfoot{}
\cfoot{}
\rfoot{\small\thepage}
\headsep 1.5em
% -----------------------------------
\usepackage{titlesec}
\renewcommand\thesubsection{(\arabic{section}.\alph{subsection})}
\titleformat{\subsection}[runin]
        {\normalfont\bfseries}
        {\thesubsection}% the label and number
        {0.5em}% space between label/number and subsection title
        {}% formatting commands applied just to subsection title
        []% punctuation or other commands following subsection title
% -----------------------------------
\setlength{\parindent}{0.0in}
\setlength{\parskip}{0.1in}
% -----------------------------------
\newcommand{\de}{\mathrm{d}}
\newcommand{\DD}{\mathrm{D}}
\newcommand{\pe}{\partial}
\newcommand{\mcal}{\mathcal}
%\newcommand{\pdx}{\left|\frac{\partial}{\partial_x}\right|}

\newcommand{\dsp}{\displaystyle}

\newcommand{\norm}[1]{\left\Vert #1 \right\Vert}
%\newcommand{\mean}[1]{\left\langle #1 \right\rangle}
\newcommand{\mean}[1]{\overline{#1}}
\newcommand{\inner}[2]{\left\langle #1,#2\right\rangle}

\newcommand{\ve}[1]{\boldsymbol{#1}}

\newcommand{\thus}{\Rightarrow \quad }
\newcommand{\fff}{\iff\quad}
\newcommand{\qdt}[1]{\quad \mbox{#1} \quad}

\renewcommand{\Re}{\mathrm{Re}}
\renewcommand{\Im}{\mathrm{Im}}
\newcommand{\E}{\mathbb{E}}
\newcommand{\lap} {\nabla^2}
\renewcommand{\div}{\nabla\cdot}

\newcommand{\csch}{\text{csch}}
\newcommand{\sech}{\text{sech}}


\newcommand{\hot}{\text{h.o.t.}}

\newcommand{\ssp}{\left.\qquad\right.}

\newcommand{\var}{\text{var}}
\newcommand{\cov}{\text{cov}}

%%%%%%%%%%%%%%%%%%%%%%%%%%%%%%%%%%%%%%%%%%%%%%%%%%
\makeatletter
\newcommand*{\mint}[1]{%
  % #1: overlay symbol
  \mint@l{#1}{}%
}
\newcommand*{\mint@l}[2]{%
  % #1: overlay symbol
  % #2: limits
  \@ifnextchar\limits{%
    \mint@l{#1}%
  }{%
    \@ifnextchar\nolimits{%
      \mint@l{#1}%
    }{%
      \@ifnextchar\displaylimits{%
        \mint@l{#1}%
      }{%
        \mint@s{#2}{#1}%
      }%
    }%
  }%
}
\newcommand*{\mint@s}[2]{%
  % #1: limits
  % #2: overlay symbol
  \@ifnextchar_{%
    \mint@sub{#1}{#2}%
  }{%
    \@ifnextchar^{%
      \mint@sup{#1}{#2}%
    }{%
      \mint@{#1}{#2}{}{}%
    }%
  }%
}
\def\mint@sub#1#2_#3{%
  \@ifnextchar^{%
    \mint@sub@sup{#1}{#2}{#3}%
  }{%
    \mint@{#1}{#2}{#3}{}%
  }%
}
\def\mint@sup#1#2^#3{%
  \@ifnextchar_{%
    \mint@sup@sub{#1}{#2}{#3}%
  }{%
    \mint@{#1}{#2}{}{#3}%
  }%
}
\def\mint@sub@sup#1#2#3^#4{%
  \mint@{#1}{#2}{#3}{#4}%
}
\def\mint@sup@sub#1#2#3_#4{%
  \mint@{#1}{#2}{#4}{#3}%
}
\newcommand*{\mint@}[4]{%
  % #1: \limits, \nolimits, \displaylimits
  % #2: overlay symbol: -, =, ...
  % #3: subscript
  % #4: superscript
  \mathop{}%
  \mkern-\thinmuskip
  \mathchoice{%
    \mint@@{#1}{#2}{#3}{#4}%
        \displaystyle\textstyle\scriptstyle
  }{%
    \mint@@{#1}{#2}{#3}{#4}%
        \textstyle\scriptstyle\scriptstyle
  }{%
    \mint@@{#1}{#2}{#3}{#4}%
        \scriptstyle\scriptscriptstyle\scriptscriptstyle
  }{%
    \mint@@{#1}{#2}{#3}{#4}%
        \scriptscriptstyle\scriptscriptstyle\scriptscriptstyle
  }%
  \mkern-\thinmuskip
  \int#1%
  \ifx\\#3\\\else_{#3}\fi
  \ifx\\#4\\\else^{#4}\fi  
}
\newcommand*{\mint@@}[7]{%
  % #1: limits
  % #2: overlay symbol
  % #3: subscript
  % #4: superscript
  % #5: math style
  % #6: math style for overlay symbol
  % #7: math style for subscript/superscript
  \begingroup
    \sbox0{$#5\int\m@th$}%
    \sbox2{$#5\int_{}\m@th$}%
    \dimen2=\wd0 %
    % => \dimen2 = width of \int
    \let\mint@limits=#1\relax
    \ifx\mint@limits\relax
      \sbox4{$#5\int_{\kern1sp}^{\kern1sp}\m@th$}%
      \ifdim\wd4>\wd2 %
        \let\mint@limits=\nolimits
      \else
        \let\mint@limits=\limits
      \fi
    \fi
    \ifx\mint@limits\displaylimits
      \ifx#5\displaystyle
        \let\mint@limits=\limits
      \fi
    \fi
    \ifx\mint@limits\limits
      \sbox0{$#7#3\m@th$}%
      \sbox2{$#7#4\m@th$}%
      \ifdim\wd0>\dimen2 %
        \dimen2=\wd0 %
      \fi
      \ifdim\wd2>\dimen2 %
        \dimen2=\wd2 %
      \fi
    \fi
    \rlap{%
      $#5%
        \vcenter{%
          \hbox to\dimen2{%
            \hss
            $#6{#2}\m@th$%
            \hss
          }%
        }%
      $%
    }%
  \endgroup
}

\begin{document}

% \section{} [Exercise 4.3.5 of \cite{Lebl_23}] 
% If $f:[0,1]\to\mathbb{R}$ has $n+1$ continuous derivatives and $x_0\in[a,b]$, prove 
% \begin{align}
%     \lim_{x\to x_0}\frac{R^{x_0}_n(x)}{(x-x_0)^n}=0.
% \end{align}

% \section{} [Exercise 5.4.11 of \cite{Lebl_23}] 
% This is an example of infinitely differentiable function that is not analytic.

% Since $(e^x)'=e^x$, it is easy to see that $e^x$ is infinitely differentiable (has derivatives of all orders). Define the function $f:\mathbb{R}\to\mathbb{R}$:
% \begin{align}
%     f(x):= \begin{cases}
%         e^{-1/x} \qdt{if} x>0\\
%         0 \qdt{if} x\leq 0
%     \end{cases}
% \end{align}

% \subsection{}
% Prove that for every $m\in\mathbb{N}$
% \begin{align}
%     \lim_{x\to 0^+} \frac{e^{-1/x}}{x^m}=0.
% \end{align}

% \subsection{}
% Prove that $f$ is infinitely differentiable.

% \subsection{}
% Compute the Taylor series for $f$ at the origin, that is,
% \begin{align}
%     \sum_{k=0}^\infty \frac{f^{(k)(0)}}{k!} x^k.
% \end{align}
% Show that it converges, but show that it does not converge to $f(x)$ for any given $x>0$.

% \section{} [Exercise 5.1.2 of \cite{Lebl_23}] 
% Let $f:[0,1]\to\mathbb{R}$ be defined by $f(x):= x$. Show that $f\in\mathcal{R}([0,1])$ and compute $\int^1_0 f$ using the definition of the integral.

\section{} [Exercise 5.1.3 of \cite{Lebl_23}] 
Let $f:[a,b]\to\mathbb{R}$ be a bounded function. Suppose there exists a sequence of partitions $\{P_k\}^\infty_{k=1}$ of $[a,b]$ such that
\begin{align}
    \lim_{k\to\infty} (U(P_k,f)-L(P_k,f)) = 0.
\end{align}
Show that $f$ is Riemann integrable and that
\begin{align}
    \int^b_a f = \lim_{k\to\infty}U(P_k,f) = \lim_{k\to\infty}L(P_k,f).
\end{align}

\section{} [Example 5.1.14 of \cite{Lebl_23}] 
Let us show $\frac{1}{1+x}$ is integrable on $[0,b]$ for all $b>0$.

% \section{} [Exercise 5.1.7 of \cite{Lebl_23}] 

\section{} [Exercise 5.2.14(a) of \cite{Lebl_23}] 
Let $f:[a,b]\to\mathbb{R}$ be a  function. Show that if $f$ is increasing, then it is Riemann integrable.

\section{} [Exercise 5.1.14 of \cite{Lebl_23}] 
Construct functions $f$ and $g$, where $f:[0,1]\to \mathbb{R}$ is Riemann integrable, $g : [0, 1] \to [0, 1]$ is one-to-one and onto, and such that the composition $f\circ g$ is not Riemann integrable.

\section{} [Exercise 5.2.5 of \cite{Lebl_23}] 
Let $f:[a,b]\to\mathbb{R}$ be a continuous function such that $f(x)\geq 0$ for all $x\in[a,b]$ and $\int^b_a f = 0$. Prove that $f(x)=0$ for all $x$.

\section{} [Exercise 5.2.10 of \cite{Lebl_23}] 
Suppose $f:[a,b]\to\mathbb{R}$ is bounded and has finitely many discontinuities. Show that as a function of $x$ the expression $|f(x)|$ is bounded with finitely many discontinuities and is thus Riemann integrable. Then show
\begin{align}
    \left| \int^b_a f(x)\;\de x \right| \leq \int^b_a |f(x)|\;\de x.
\end{align}



\section{} [Exercise 5.2.4 of \cite{Lebl_23}] 
Prove the mean value theorem for integrals: If $f:[a,b]\to\mathbb{R}$ is continuous, then there exists a $c\in[a,b]$ such that
\begin{align}
    \int^b_a f = f(c)(b-a).
\end{align}

\section{} [Exercise 5.3.10 of \cite{Lebl_23}] 
A function $f$ is an odd function if $f(x)=-f(-x)$, and $f$ is an even function if $f(x)=f(-x)$. Let $a> 0$. Assume $f$ is continuous. Prove:

\subsection{}
If $f$ is odd, then $\int^a_{-a} f = 0$.

\subsection{}
If $f$ is even, then $\int^a_{-a} f = 2\int^a_{0} f$.


\vfill
\printbibliography


\end{document}