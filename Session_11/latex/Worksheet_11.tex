\documentclass[11pt,letterpaper]{article}
\usepackage[utf8]{inputenc}
\usepackage[left=1in,right=1in,top=1in,bottom=1in]{geometry}
\usepackage{amsfonts,amsmath,amsthm}
\usepackage{graphicx,float}
% -----------------------------------
\usepackage{hyperref}
\hypersetup{%
  colorlinks=true,
  linkcolor=blue,
  citecolor=blue,
  urlcolor=blue,
  linkbordercolor={0 0 1}
}
% -----------------------------------
\usepackage[style=authoryear,backend=biber]{biblatex}
\addbibresource{citation.bib}
% -----------------------------------
\usepackage{fancyhdr}
\newcommand\course{MATH-UA.0325\\Analysis}
\newcommand\hwnumber{11}                  % <-- homework number
\newcommand\NetIDa{Ryan Sh\`iji\'e D\`u} 
\newcommand\NetIDb{April 19th, 2024}
\pagestyle{fancyplain}
\headheight 35pt
\lhead{\NetIDa\\\NetIDb}
\chead{\textbf{\Large Worksheet \hwnumber}}
\rhead{\course}
\lfoot{}
\cfoot{}
\rfoot{\small\thepage}
\headsep 1.5em
% -----------------------------------
\usepackage{titlesec}
\renewcommand\thesubsection{(\arabic{section}.\alph{subsection})}
\titleformat{\subsection}[runin]
        {\normalfont\bfseries}
        {\thesubsection}% the label and number
        {0.5em}% space between label/number and subsection title
        {}% formatting commands applied just to subsection title
        []% punctuation or other commands following subsection title
% -----------------------------------
\setlength{\parindent}{0.0in}
\setlength{\parskip}{0.1in}
% -----------------------------------
\input{../../command.tex}
\begin{document}

\section{} [Exercise 4.3.5 of \cite{Lebl_23}] 
If $f:[0,1]\to\mathbb{R}$ has $n+1$ continuous derivatives and $x_0\in[a,b]$, prove 
\begin{align}
    \lim_{x\to x_0}\frac{R^{x_0}_n(x)}{(x-x_0)^n}=0.
\end{align}

\section{} [Exercise 5.4.11 of \cite{Lebl_23}] 
This is an example of infinitely differentiable function that is not analytic.

Since $(e^x)'=e^x$, it is easy to see that $e^x$ is infinitely differentiable (has derivatives of all orders). Define the function $f:\mathbb{R}\to\mathbb{R}$:
\begin{align}
    f(x):= \begin{cases}
        e^{-1/x} \qdt{if} x>0\\
        0 \qdt{if} x\leq 0
    \end{cases}
\end{align}

\subsection{}
Prove that for every $m\in\mathbb{N}$
\begin{align}
    \lim_{x\to 0^+} \frac{e^{-1/x}}{x^m}=0.
\end{align}

\subsection{}
Prove that $f$ is infinitely differentiable.

\subsection{}
Compute the Taylor series for $f$ at the origin, that is,
\begin{align}
    \sum_{k=0}^\infty \frac{f^{(k)(0)}}{k!} x^k.
\end{align}
Show that it converges, but show that it does not converge to $f(x)$ for any given $x>0$.

\section{} [Exercise 5.1.2 of \cite{Lebl_23}] 
Let $f:[0,1]\to\mathbb{R}$ be defined by $f(x):= x$. Show that $f\in\mathcal{R}([0,1])$ and compute $\int^1_0 f$ using the definition of the integral.

% \section{} [Exercise 5.1.3 of \cite{Lebl_23}] 
% Let $f:[a,b]\to\mathbb{R}$ be a bounded function. Suppose there exists a sequence of partitions $\{P_k\}^\infty_{k=1}$ of $[a,b]$ such that
% \begin{align}
%     \lim_{k\to\infty} (U(P_k,f)-L(P_k,f)) = 0.
% \end{align}
% Show that $f$ is Riemann integrable and that
% \begin{align}
%     \int^b_a f = \lim_{k\to\infty}U(P_k,f) = \lim_{k\to\infty}L(P_k,f).
% \end{align}

% \section{} [Exercise 5.1.7 of \cite{Lebl_23}] 

% \section{} [Exercise 5.2.5 of \cite{Lebl_23}] 
% Let $f:[a,b]\to\mathbb{R}$ be a continuous function such that $f(x)\geq 0$ for all $x\in[a,b]$ and $\int^b_a f = 0$. Prove that $f(x)=0$ for all $x$.

\section{} [Exercise 5.2.10 of \cite{Lebl_23}] 
Suppose $f:[a,b]\to\mathbb{R}$ is bounded and has finitely many discontinuities. Show that as a function of $x$ the expression $|f(x)|$ is bounded with finitely many discontinuities and is thus Riemann integrable. Then show
\begin{align}
    \left| \int^b_a f(x)\;\de x \right| \leq \int^b_a |f(x)|\;\de x.
\end{align}

\section{} [Exercise 5.2.14(a) of \cite{Lebl_23}] 
Let $f:[a,b]\to\mathbb{R}$ be a  function. Show that if $f$ is increasing, then it is Riemann integrable.

\section{} [Exercise 5.3.10 of \cite{Lebl_23}] 
A function $f$ is an odd function if $f(x)=-f(-x)$, and $f$ is an even function if $f(x)=f(-x)$. Let $a> 0$. Assume $f$ is continuous. Prove:

\subsection{}
If $f$ is odd, then $\int^a_{-a} f = 0$.

\subsection{}
If $f$ is even, then $\int^a_{-a} f = 2\int^a_{0} f$.

% \section{} [Exercise 5.2.4 of \cite{Lebl_23}] 
% Prove the mean value theorem for integrals: If $f:[a,b]\to\mathbb{R}$ is continuous, then there exists a $c\in[a,b]$ such that
% \begin{align}
%     \int^b_a f = f(c)(b-a).
% \end{align}


\vfill
\printbibliography


\end{document}