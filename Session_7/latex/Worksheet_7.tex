\documentclass[11pt,letterpaper]{article}
\usepackage[utf8]{inputenc}
\usepackage[left=1in,right=1in,top=1in,bottom=1in]{geometry}
\usepackage{amsfonts,amsmath,amsthm}
\usepackage{graphicx,float}
% -----------------------------------
\usepackage{hyperref}
\hypersetup{%
  colorlinks=true,
  linkcolor=blue,
  citecolor=blue,
  urlcolor=blue,
  linkbordercolor={0 0 1}
}
% -----------------------------------
\usepackage[style=authoryear,backend=biber]{biblatex}
\addbibresource{citation.bib}
% -----------------------------------
\usepackage{fancyhdr}
\newcommand\course{MATH-UA.0325\\Analysis}
\newcommand\hwnumber{7}                  % <-- homework number
\newcommand\NetIDa{Ryan Sh\`iji\'e D\`u} 
\newcommand\NetIDb{October 24th, 2024}
\pagestyle{fancyplain}
\headheight 35pt
\lhead{\NetIDa\\\NetIDb}
\chead{\textbf{\Large Worksheet \hwnumber}}
\rhead{\course}
\lfoot{}
\cfoot{}
\rfoot{\small\thepage}
\headsep 1.5em
% -----------------------------------
\usepackage{titlesec}
\renewcommand\thesubsection{(\arabic{section}.\alph{subsection})}
\titleformat{\subsection}[runin]
        {\normalfont\bfseries}
        {\thesubsection}% the label and number
        {0.5em}% space between label/number and subsection title
        {}% formatting commands applied just to subsection title
        []% punctuation or other commands following subsection title
% -----------------------------------
\setlength{\parindent}{0.0in}
\setlength{\parskip}{0.1in}
% -----------------------------------
\input{../../command.tex}
\begin{document}

\section{} 
\subsection{}[Exercise 3.1.9 of \cite{Lebl_23}] 
Let $c_1$ be a cluster point of $A\subset\mathbb{R}$ and $c_2$ be a cluster point of $B\subset\mathbb{R}$. Suppose $f:A\to B$
and $g:B\to \mathbb{R}$ are functions such that $f(x)\to c_2$ as $x\to c_1$ and $g(y)\to L$ as $y\to c_2$. 

Let $h(x):= g(f(x))$, we have ``chain rule'' for limits: $h(x)\to L$ as $x\to c_1$, if we also suppose that $g(c_2)=L$ (that is, $g(x)$ is continuous at $c_2$).

\subsection{}[Exercise 3.1.14 of \cite{Lebl_23}] 
Show via a counterexample that the assumption of $g(c_2)=L$ is necessary.

\section{} [Exercise 3.2.1 of \cite{Lebl_23}] 
Using the definition of continuity directly prove that $f: \mathbb{R}\to \mathbb{R}$ defined by $f(x) := x^2$ is continuous.

\section{} [Example 3.2.6 of \cite{Lebl_23}] 
Show that the functions $\sin(x)$ and $\cos(x)$ are continuous. \\
Hint: use the sum-to-product trigonometric identities.

\section{} [Example 3.2.12 of \cite{Lebl_23}] 
Show that the popcorn function (or the Thomae function) is continuous for all irrational numbers and discontinuous for all rational numbers. The popcorn function is defined as $f:(0,1)\to \mathbb{R}$
\begin{align}
    f(x) := \begin{cases}
        1/k & \text{if } x=m/k, \text{ where }m,k\in\mathbb{N}\text{ and have no common divisors,}\\
        0 & \text{if }x\text{ is irrational.}
    \end{cases}
\end{align}

\begin{figure}[H]
    \centering
    \includegraphics[width=0.5\linewidth]{Thomae_function}
    \caption{Graph of the popcorn function. }
\end{figure}

\section{} [Exercise 3.2.4 and 3.2.4 of \cite{Lebl_23}] 
Is $f: \mathbb{R}\to \mathbb{R}$ continuous? Prove your assertion.

\subsection{}
\begin{align}
    f(x) := \begin{cases}
        \sin(1/x) &\text{if } x\neq 0,\\
        0  &\text{if } x= 0.
    \end{cases}
\end{align}

\subsection{}
\begin{align}
    f(x) := \begin{cases}
        x\sin(1/x) &\text{if } x\neq 0,\\
        0  &\text{if } x= 0.
    \end{cases}
\end{align}

\section{} [Exercise 3.2.10 of \cite{Lebl_23}] 
Let $f:\mathbb{R}\to \mathbb{R}$ and $g:\mathbb{R}\to \mathbb{R}$ be continuous functions. Suppose that $f(r)=g(r)$ for all $r\in\mathbb{Q}$. Show that $f(x)=g(x)$ for all $x\in\mathbb{R}$. 


\vfill
\printbibliography


\end{document}