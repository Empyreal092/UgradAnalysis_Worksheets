\documentclass[11pt,letterpaper]{article}
\usepackage[utf8]{inputenc}
\usepackage[left=1in,right=1in,top=1in,bottom=1in]{geometry}
\usepackage{amsfonts,amsmath}
\usepackage{graphicx,float}
% -----------------------------------
\usepackage{hyperref}
\hypersetup{%
  colorlinks=true,
  linkcolor=blue,
  citecolor=blue,
  urlcolor=blue,
  linkbordercolor={0 0 1}
}
% -----------------------------------
\usepackage{fancyhdr}
\newcommand\course{MATH-UA.0325\\Analysis}
\newcommand\hwnumber{1}                  % <-- homework number
\newcommand\NetIDa{Ryan Sh\`iji\'e D\`u} 
\newcommand\NetIDb{February 2nd, 2024}
\pagestyle{fancyplain}
\headheight 35pt
\lhead{\NetIDa\\\NetIDb}
\chead{\textbf{\Large Worksheet \hwnumber}}
\rhead{\course}
\lfoot{}
\cfoot{}
\rfoot{\small\thepage}
\headsep 1.5em
% -----------------------------------
\usepackage{titlesec}
\renewcommand\thesubsection{(\arabic{section}.\alph{subsection})}
\titleformat{\subsection}[runin]
        {\normalfont\bfseries}
        {\thesubsection}% the label and number
        {0.5em}% space between label/number and subsection title
        {}% formatting commands applied just to subsection title
        []% punctuation or other commands following subsection title
% -----------------------------------
\setlength{\parindent}{0.0in}
\setlength{\parskip}{0.1in}
% -----------------------------------
\input{../../command.tex}
\begin{document}

\section{}
Show that the following two statements are equivalent:

\subsection{}
\begin{enumerate}
    \item $f: B\to C$ is injective.
    \item For any set $A$ and any two functions $g,h: A\to B$ such that $f\circ g = f\circ h$, we must have $g=h$. 
\end{enumerate}

\subsection{}
\begin{enumerate}
    \item $f: A\to B$ is surjective.
    \item For any set $C$ and any two functions $g,h: B\to C$ such that $g\circ f = h\circ f$, we must have $g=h$. 
\end{enumerate}

\section{}
Let $f: A\to B$ and $g: B\to C$ be two bijective functions.  Show $g\circ f$ is bijective.

\section{}
Let $f: A\to B$, $C, D\subset A$, and $P, Q\subset B$. In lecture, you learned 
\begin{align}
    f(C\cup D) = f(C)\cup f(D).
\end{align}

Show
\begin{align}
    &f(C\cap D) \subset f(C)\cap f(D);\\
    &f^{-1}(P\cup Q) = f^{-1}(P)\cup f^{-1}(Q);\\
    &f^{-1}(P\cap Q) = f^{-1}(P)\cap f^{-1}(Q).
\end{align}

\section{}
Let $f: X\to Y$ be a function between sets. Let $A,B\subset X$ be subsets. 

\subsection{}
Show
\begin{align}
    f(A)/f(B) \subset f(A/B).
\end{align}

\subsection{}
Give an example function that shows the other direction is false.

\section{}
Let $f: X\to Y$ be a function between sets and $U\subset A$ and $V\subset B$. 

\subsection{}
Show that
\begin{align}
    f(f^{-1}(V)) \subset V \qdt{and} U \subset f^{-1}(f(U)).
\end{align}

\subsection{}
Give an example that shows the above relation cannot be equality.

\section{}
If two sets $A$ and $B$ have the same cardinally, we write $A\sim B$. Show that $\sim$ is an equivalence relation on sets.

\section{}
Decide for which natural numbers the inequality $2n > n^2$ is true.
Prove your claim using induction.

    
    
% \bibliographystyle{alpha}
% \bibliography{citation}


\end{document}