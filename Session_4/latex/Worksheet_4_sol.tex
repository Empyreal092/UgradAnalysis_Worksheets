\documentclass[11pt,letterpaper]{article}
\usepackage[utf8]{inputenc}
\usepackage[left=1in,right=1in,top=1in,bottom=1in]{geometry}
\usepackage{amsfonts,amsmath,amsthm}
\usepackage{graphicx,float}
% -----------------------------------
\usepackage{hyperref}
\hypersetup{%
  colorlinks=true,
  linkcolor=blue,
  citecolor=blue,
  urlcolor=blue,
  linkbordercolor={0 0 1}
}
% -----------------------------------
\usepackage[style=authoryear-icomp,backend=biber]{biblatex}
\addbibresource{citation.bib}
% -----------------------------------
\usepackage{fancyhdr}
\newcommand\course{MATH-UA.0325\\Analysis}
\newcommand\hwnumber{4 Solutions}                  % <-- homework number
\newcommand\NetIDa{Ryan Sh\`iji\'e D\`u} 
\newcommand\NetIDb{February 23rd, 2024}
\pagestyle{fancyplain}
\headheight 35pt
\lhead{\NetIDa\\\NetIDb}
\chead{\textbf{\Large Worksheet \hwnumber}}
\rhead{\course}
\lfoot{}
\cfoot{}
\rfoot{\small\thepage}
\headsep 1.5em
% -----------------------------------
\usepackage{titlesec}
\renewcommand\thesubsection{(\arabic{section}.\alph{subsection})}
\titleformat{\subsection}[runin]
        {\normalfont\bfseries}
        {\thesubsection}% the label and number
        {0.5em}% space between label/number and subsection title
        {}% formatting commands applied just to subsection title
        []% punctuation or other commands following subsection title
% -----------------------------------
\setlength{\parindent}{0.0in}
\setlength{\parskip}{0.1in}
% -----------------------------------
\newcommand{\de}{\mathrm{d}}
\newcommand{\DD}{\mathrm{D}}
\newcommand{\pe}{\partial}
\newcommand{\mcal}{\mathcal}
%\newcommand{\pdx}{\left|\frac{\partial}{\partial_x}\right|}

\newcommand{\dsp}{\displaystyle}

\newcommand{\norm}[1]{\left\Vert #1 \right\Vert}
%\newcommand{\mean}[1]{\left\langle #1 \right\rangle}
\newcommand{\mean}[1]{\overline{#1}}
\newcommand{\inner}[2]{\left\langle #1,#2\right\rangle}

\newcommand{\ve}[1]{\boldsymbol{#1}}

\newcommand{\thus}{\Rightarrow \quad }
\newcommand{\fff}{\iff\quad}
\newcommand{\qdt}[1]{\quad \mbox{#1} \quad}

\renewcommand{\Re}{\mathrm{Re}}
\renewcommand{\Im}{\mathrm{Im}}
\newcommand{\E}{\mathbb{E}}
\newcommand{\lap} {\nabla^2}
\renewcommand{\div}{\nabla\cdot}

\newcommand{\csch}{\text{csch}}
\newcommand{\sech}{\text{sech}}


\newcommand{\hot}{\text{h.o.t.}}

\newcommand{\ssp}{\left.\qquad\right.}

\newcommand{\var}{\text{var}}
\newcommand{\cov}{\text{cov}}

%%%%%%%%%%%%%%%%%%%%%%%%%%%%%%%%%%%%%%%%%%%%%%%%%%
\makeatletter
\newcommand*{\mint}[1]{%
  % #1: overlay symbol
  \mint@l{#1}{}%
}
\newcommand*{\mint@l}[2]{%
  % #1: overlay symbol
  % #2: limits
  \@ifnextchar\limits{%
    \mint@l{#1}%
  }{%
    \@ifnextchar\nolimits{%
      \mint@l{#1}%
    }{%
      \@ifnextchar\displaylimits{%
        \mint@l{#1}%
      }{%
        \mint@s{#2}{#1}%
      }%
    }%
  }%
}
\newcommand*{\mint@s}[2]{%
  % #1: limits
  % #2: overlay symbol
  \@ifnextchar_{%
    \mint@sub{#1}{#2}%
  }{%
    \@ifnextchar^{%
      \mint@sup{#1}{#2}%
    }{%
      \mint@{#1}{#2}{}{}%
    }%
  }%
}
\def\mint@sub#1#2_#3{%
  \@ifnextchar^{%
    \mint@sub@sup{#1}{#2}{#3}%
  }{%
    \mint@{#1}{#2}{#3}{}%
  }%
}
\def\mint@sup#1#2^#3{%
  \@ifnextchar_{%
    \mint@sup@sub{#1}{#2}{#3}%
  }{%
    \mint@{#1}{#2}{}{#3}%
  }%
}
\def\mint@sub@sup#1#2#3^#4{%
  \mint@{#1}{#2}{#3}{#4}%
}
\def\mint@sup@sub#1#2#3_#4{%
  \mint@{#1}{#2}{#4}{#3}%
}
\newcommand*{\mint@}[4]{%
  % #1: \limits, \nolimits, \displaylimits
  % #2: overlay symbol: -, =, ...
  % #3: subscript
  % #4: superscript
  \mathop{}%
  \mkern-\thinmuskip
  \mathchoice{%
    \mint@@{#1}{#2}{#3}{#4}%
        \displaystyle\textstyle\scriptstyle
  }{%
    \mint@@{#1}{#2}{#3}{#4}%
        \textstyle\scriptstyle\scriptstyle
  }{%
    \mint@@{#1}{#2}{#3}{#4}%
        \scriptstyle\scriptscriptstyle\scriptscriptstyle
  }{%
    \mint@@{#1}{#2}{#3}{#4}%
        \scriptscriptstyle\scriptscriptstyle\scriptscriptstyle
  }%
  \mkern-\thinmuskip
  \int#1%
  \ifx\\#3\\\else_{#3}\fi
  \ifx\\#4\\\else^{#4}\fi  
}
\newcommand*{\mint@@}[7]{%
  % #1: limits
  % #2: overlay symbol
  % #3: subscript
  % #4: superscript
  % #5: math style
  % #6: math style for overlay symbol
  % #7: math style for subscript/superscript
  \begingroup
    \sbox0{$#5\int\m@th$}%
    \sbox2{$#5\int_{}\m@th$}%
    \dimen2=\wd0 %
    % => \dimen2 = width of \int
    \let\mint@limits=#1\relax
    \ifx\mint@limits\relax
      \sbox4{$#5\int_{\kern1sp}^{\kern1sp}\m@th$}%
      \ifdim\wd4>\wd2 %
        \let\mint@limits=\nolimits
      \else
        \let\mint@limits=\limits
      \fi
    \fi
    \ifx\mint@limits\displaylimits
      \ifx#5\displaystyle
        \let\mint@limits=\limits
      \fi
    \fi
    \ifx\mint@limits\limits
      \sbox0{$#7#3\m@th$}%
      \sbox2{$#7#4\m@th$}%
      \ifdim\wd0>\dimen2 %
        \dimen2=\wd0 %
      \fi
      \ifdim\wd2>\dimen2 %
        \dimen2=\wd2 %
      \fi
    \fi
    \rlap{%
      $#5%
        \vcenter{%
          \hbox to\dimen2{%
            \hss
            $#6{#2}\m@th$%
            \hss
          }%
        }%
      $%
    }%
  \endgroup
}

\begin{document}

\section{}
Let $\{a_n\}_{n\geq 1}$ be a Cauchy sequence of real numbers. Show that
$\{a_n^2\}_{n\geq 1}$ is also a Cauchy sequence.

\begin{proof}
Since $\{a_n\}_{n\geq 1}$ is Cauchy, we know that it is bounded by $A\in\mathbb{R}$. Additionally, for all $\epsilon>0$ there exists $N_{\epsilon}\in \mathbb{N}$ s.t.
\begin{align*}
    |a_n-a_m|<\frac{\epsilon}{2A} \qdt{for all} n,m\geq N_\epsilon.
\end{align*}
Then we have for all $n,m \geq N_{\epsilon}$
\begin{align*}
    |a_n^2-a_m^2|=|a_n+a_m||a_n-a_m|< 2A\cdot \frac{\epsilon}{2A} = \epsilon
\end{align*}
This work for all $\epsilon>0$ so the sequence $\{a_n^2\}_{n\geq 1}$ is also Cauchy. 
\end{proof}

\section{}
Let $\{a_n\}_{n\geq 1}$ be a sequence defined by the following rule:
\begin{align}
    a_1 = 3 \qdt{and} a_{n+1}=\frac{a_n}{2}+\frac{1}{a_n} \qdt{for all} n\geq 1.
\end{align}

\subsection{} Show that the sequence is bounded below by $\sqrt{2}$.
\begin{proof}
It is clear that the sequence is positive. Then we see 
\begin{align*}
    a_{n+1}-\sqrt{2} = \frac{a_n}{2}+\frac{1}{a_n}-\sqrt{2} = \frac{a_n^2+2-2\sqrt{2}a_n}{2a_n} = \frac{(a_n-\sqrt{2})^2}{2a_n} >0.
\end{align*}
Therefore $a_n>\sqrt{2}$ for all $n$.
\end{proof}

\subsection{} Show that this is a sequence of rational numbers.
\begin{proof}
The set $\mathbb{Q}$ is a field and has the closure property. Therefore $a_n\in\mathbb{Q}$ for all $n$.
\end{proof}

\subsection{} Prove that the sequence is monotonically decreasing.
\begin{proof}
For all $n\in \mathbb{N}$
\begin{align*}
    a_{n+1}-a_n = \frac{-a_n}{2}+\frac{1}{a_n} = \frac{-a_n^2+2}{2a_n}<0
\end{align*}
where we used the fact that $a_n > \sqrt{2}$. $\{a_n\}_{n\in \mathbb{N}}$ is monotonically decreasing.
\end{proof}

\subsection{} Deduce that $\{a_n\}_{n\geq 1}$ converges and find its limit.
\begin{proof}
$\{a_n\}_{n\in \mathbb{N}}$ is bounded and monotonically decreasing therefore it converges.

We name $\lim_{n\rightarrow \infty}a_n = a$. By the recurrence relation $a$ needs to satisfy
\begin{align}
    a=\frac{a}{2}+\frac{1}{a}
\end{align}
which has positive solution $\sqrt{2}$. Thus, $\lim_{n\rightarrow \infty}a_n = \sqrt{2}$.
\end{proof}

Remark: This is an example of Cauchy sequence of rational numbers
converging to an irrational number.

\section{}
Let $\{a_n\}_{n\geq 1}$ and $\{b_n\}_{n\geq 1}$ be two bounded sequences. Show that
\begin{align}
    \limsup_{n\to\infty} (a_n+b_n) \leq \limsup_{n\to\infty} a_n+\limsup_{n\to\infty} b_n.
\end{align}

\begin{proof}
By definition, 
\begin{align*}
    \limsup_{n\rightarrow\infty}(a_n+b_n) &= \lim_{N\rightarrow\infty} \sup \{(a_n+b_n), n\geq N\}\\
    \limsup_{n\rightarrow\infty}a_n &= \lim_{N\rightarrow\infty} \sup \{a_n, n\geq N\}\\
    \limsup_{n\rightarrow\infty}b_n &= \lim_{N\rightarrow\infty} \sup \{b_n, n\geq N\}.
\end{align*}
We name $A_N = \sup \{a_n, n\geq N\}$ and $B_N = \sup \{b_n, n\geq N\}$. Then for all $n\geq N$, $A_N\geq a_n$, $B_N\geq b_n$, and thus $A_N+B_N\geq a_n+b_n$. $A_N+B_N$ is an upper bound of $\{a_n+b_n, n\geq N\}$. \\
Since $\sup \{a_n+b_n, n\geq N\}$ is the least upper bound 
\begin{align*}
    &A_N+B_N\geq \sup \{a_n+b_n, n\geq N\}\\
    \thus &\sup \{a_n, n\geq N\}+\sup \{b_n, n\geq N\} \geq \sup \{a_n+b_n, n\geq N\}.
\end{align*}

We can take the limit to infinity of both sides and we have 
$$\lim_{N\rightarrow\infty} \sup \{(a_n+b_n), n\geq N\}\leq \lim_{N\rightarrow\infty} \left(\sup \{a_n, n\geq N\}+\sup \{b_n, n\geq N\}\right).$$ Because both $\sup$ are bounded 
\begin{align*}
    &\lim_{N\rightarrow\infty} \left(\sup \{a_n, n\geq N\}+\sup \{b_n, n\geq N\}\right)=\lim_{N\rightarrow\infty} \sup \{a_n, n\geq N\}+\lim_{N\rightarrow\infty} \sup \{b_n, n\geq N\} \\
    \thus &\limsup_{n\rightarrow\infty}(a_n+b_n) \leq \limsup_{n\rightarrow\infty}a_n+\limsup_{n\rightarrow\infty}b_n.
\end{align*}
\end{proof}

% \section{}
% Let $\{a_n\}_{n\geq 1}$ and $\{b_n\}_{n\geq 1}$ be two bounded sequences of non-negative numbers. Show that
% \begin{align}
%     \limsup_{n\to\infty} (a_nb_n) \leq (\limsup_{n\to\infty} a_n)(\limsup_{n\to\infty} b_n).
% \end{align}

\section{}
Let $\{a_n\}_{n\geq 1}$ be a sequence of real numbers that is bounded above.
Prove that $L = \limsup a_n$ has the following properties:

\subsection{} \label{sec:4a}
For every $\epsilon > 0$ there are only finitely many $n$ for which $a_n>L+\epsilon$.

\begin{proof}
Name $v_N = \sup\{a_n, n\geq N\}$. Then by definition $L = \limsup a_n = \lim_{N\rightarrow \infty}v_N$. We also know that if $\{v_N\}_{N\geq 1}$ is a decreasing sequence and $v_n\geq L$.

For all $\epsilon>0$, there exists $N\in \mathbb{N}$ s.t. for all $n\geq N$, 
\begin{align*}
    &|v_n-L|<\epsilon \\
    \thus &v_n<L+\epsilon\\
    \thus &\sup\{a_m, m\geq n\}<L+\epsilon\\
    \thus &a_n<L+\epsilon.
\end{align*}
This leaves only some or all $a_n, n<N$ (only finitely amount) that could have
$$a_n>L+\epsilon.$$
\end{proof}

\subsection{} For every $\epsilon > 0$ there are infinitely many $n$ for which $a_n>L-\epsilon$.
\begin{proof}
We know there exists a subsequence $\{a_{k_n}\}_{n\geq 1}$ that converges to $L$. Then for all $\epsilon>0$, $\exists N\in\mathbb{N}$ s.t. for all $n\geq N$
\begin{align*}
    |a_{k_n}-L|<\epsilon \Rightarrow L-\epsilon<a_{k_n}
\end{align*}
This means there are infinitely many $a_{k_n}$ thus $a_{n}$ larger than $L-\epsilon$.
\end{proof}

Remark: It is also true that there can be at most one real number with both of the above two properties.

\section{}
Show that a sequence $\{a_n\}_{n\geq 1}$ is bounded if and only if $\limsup_{n\to\infty} |a_n|<\infty$.

\begin{proof}
"$\Rightarrow$": We know 
\begin{align*}
    \limsup_{n\to\infty} |a_n| \leq \sup \{|a_n|, n\geq 1\}.
\end{align*}
However the RHS is bounded because $\{a_n\}_{n\geq 1}$ is bounded.

"$\Leftarrow$": We know from Problem \ref{sec:4a} that for $\epsilon = 1$ there are only finitely many $n$ for which $|a_n|>\limsup_{n\to\infty} |a_n|+1$. Then for some finite $N$
\begin{align}
    \sup_{n} \{|a_n|\} \leq \max\{a_1,\dots,a_N,\limsup_{n\to\infty} |a_n|+1\}.
\end{align}
This is finite because it is a maximum over finite elements. Finally, $\sup_{n} \{|a_n|\}<\infty$ implies the sequence is bounded.

\end{proof}


\vfill
\printbibliography


\end{document}